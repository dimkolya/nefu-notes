\documentclass[9pt]{article}
\usepackage{cmap}
\usepackage[T2A]{fontenc}
\usepackage[utf8]{inputenc}
\usepackage[english, russian]{babel}
\usepackage[margin=1cm,portrait]{geometry}
\usepackage{pgfplots}
\usepackage{amsmath}
\usepackage{MnSymbol}
\usepackage{wasysym}
\usepackage{tkz-euclide}
\usepackage{graphicx}
\usepackage{chngcntr}
\usepackage{wrapfig}
\usepackage{amsfonts}
\usepackage[bb=boondox]{mathalfa}
\usepackage{geometry}
\usepackage{physics}

\geometry{legalpaper, paperheight=16383pt, margin=1in}
\counterwithin*{equation}{section}
\counterwithin*{equation}{subsection}
\pagenumbering{gobble}

\DeclareSymbolFont{md}{OMX}{mdput}{m}{n} 
\DeclareMathSymbol{\intop}{\mathop}{md}{90}

\tikzset{every picture/.append style=
    {scale=3,
    axis/.style={->,blue,thick}, 
    vector/.style={-stealth,red,very thick},
    vector guide/.style={dashed,black,thick}}}

\DeclareMathOperator\Arg{Arg}
\DeclareMathOperator\sign{sign}
\DeclareMathOperator\cond{cond}
\DeclareMathOperator\const{const}
\DeclareMathOperator\gra{grad}
\DeclareMathOperator\fr{fr}
\DeclareMathOperator\Ln{Ln}
\DeclareMathOperator\di{div}
\DeclareMathOperator\rot{rot}
\DeclareMathOperator\res{res}


\begin{document}

\begin{center}
    \huge\textbf{Вариационное исчисление.}
\end{center}

\section{Введение.}

\par\ 
\par\textbf{Определение.} \textit{Функционалом} называется отображение \(J:A\to \mathbb{R}\), где \(A\) -- произвольное множество.
\par\textbf{Определение.} \textit{Линейным пространством} \(L\) называется множество, удовлетворяющее следующим свойствам:
\par1. \(\forall x, y \in L\quad\exists! z \in L:\ z = x + y\);
\par2. \(x+y=y+x,\quad x+(y+z)=(x+y)+z\) -- коммутативность и ассоциативность;
\par3. \(\exists \mathbb{0}:\ x+\mathbb{0}=x\) -- нейтральный элемент относительно <<+>>;
\par4. \(\forall x \in L \quad \exists! (-x) \in L:\ x+(-x)=\mathbb{0}\) -- противоположный элемент относительно <<+>>;
\par5. \(\forall \alpha,\beta\in\mathbb{R(C)}, x\in L:\ (\alpha\beta)x=\alpha(\beta x)\);
\par6. \(\forall \alpha,\beta\in\mathbb{R(C)}, x,y\in L:\ (\alpha+\beta)x=\alpha x + \beta x, \alpha(x+y)=\alpha x + \alpha y\).
\par\textbf{Определение.} Множество \(X\) называется \textit{метрическим}, если введена функция \(\rho: X\times X\to \mathbb{R_+}\), и она удовлетворяет трем свойствам:
\par1. \(\rho(x,y)=0 \Leftrightarrow x = y\);
\par2. \(\rho(x,y)=\rho(y,x)\) -- симметричность;
\par3. \(\rho(x,z) + \rho(z,y) \ge \rho(x,y)\) -- неравенство треугольника.
\par\textbf{Определение.} Элемент \(x\) метрики \((X, \rho)\) называется \textit{пределом последовательности} \(\{x_n\}\), если \(\displaystyle\lim_{n\to\infty}\rho(x_n,x)=0\).
\par\textbf{Определение.} Последовательность \(\{x_n\}\) из метрики \((X, \rho)\) \textit{сходится в себе}, если \[\forall\varepsilon>0\quad\exists N \in \mathbb{N}:\ \rho(x_n,x_m)<\varepsilon\quad\forall n,m>N.\]
\par\textbf{Определение.} Метрическое пространство \(X\) называется \textit{полным}, если каждая сходящаяся в себе подпоследовательность \(\{x_n\}\) из \(X\) сходится к \(x\in X\).
\par\textbf{Определение.} Линейное пространство \(E\) называется нормированным, если введена функция \(||\cdot||:E\to\mathbb{R}\) и она удовлетворяет трем свойствам:
\par1. \(||x||\ge0,\ ||x|| = 0 \Leftrightarrow x = \mathbb{0}\) -- положительная определенность;
\par2. \(||\lambda x||=|\lambda|\cdot||x||\) -- линейность;
\par3. \(||x+y||=||x||+||y||\) -- аддитивность.
\parЕсли \((X,||\cdot||)\) -- нормированное пространство, \(\rho(x,y)=||x-y||\), то \(\rho\) -- метрика в \(X\) (выполнение свойств очевидно). В этом случае говорят, что \textit{метрика \(\rho\) порождена нормой}.
\par\textbf{Определение.} Элемент \(x\) нормированного пространства \((X, ||\cdot||)\) называется \textit{пределом последовательности по норме} \(\{x_n\}\), если \(\displaystyle\lim_{n\to\infty}||x_n-x||=0\).
\par\textbf{Определение.} \textit{Пространство Банаха (Банахово пространство)} -- пространство, полное в смысле сходимости по норме. То есть каждая сходящаяся в себе подпоследовательность \(\{x_n\}\) из нормированного пространства \((X, \rho)\) сходится в себе.
\par\textbf{Определение.} Линейным пространством \textit{со скалярным произведением} называют пару \((H, (\cdot, \cdot))\), где функция \((\cdot,\cdot):H\times H \to \mathbb{R}\) -- \textit{скалярное произведение}, удовлетворяющее трем свойствам:
\par1. \((x,y)=(y,x)\) -- симметричность;
\par2. \((\lambda x + \mu y, z)=\lambda(x,z)+\mu(y,z)\) -- линейность;
\par3. \((x,x)\ge0,\ (x,x)=0\Leftrightarrow x=0\).
\par\textbf{Определение.} \textit{Гильбертово пространство} -- полное пространство со скалярным произведением. Гильбертово пространство является частным случаем Баноховых пространств.
\par\textbf{Определение.} \textit{Линейным функционалом} на нормированном пространстве \(E\) называется линейная функция \(l:E\to\mathbb{R}\).
\par\textbf{Определение.} Линейный функционал \(l:X\to\mathbb{R}\) называется \textit{непрерывным}, если для любой подпоследовательности \(\{x_n\}\) выполнено: \(x_n\to x \Leftrightarrow l(x_n)\to l(x)\). Линейный функционал \(l\) непрерывен тогда и только тогда, когда он ограничен, то есть \[\exists c>0:\ |l(x)|\le c||x||\quad\forall x \in X.\]
\parБудем рассматривать вариационные задачи для функционалов, заданных в пространствах \(C, C^{(1)}\) и \(C^{(n)}\). Пространство \(C\) состоит из функций \(f(x)\), непрерывных на отрезке \([a,b]\), а норма определяется неравенством\[||f||_C=max_{x\in[a,b]}|f(x)|.\]
\parЧерез \(C^{(1)}\) будем обозначать пространство непрерывных дифференцируемых на \([a,b]\) функций \(f(x)\) с нормой \[||f||_{C^{(1)}}=max_{x\in[a,b]}|f(x)|+max_{x\in[a,b]||f'(x)|=||f||_C+||f'||_C}.\]
\parПространство \(C^(n)\) состоит из функций \(f(x)\) имеющих на \([a,b]\) непрерывные производные \(f^(k), k = 1,2,...,n\), до \(n\)-го порядка включительно, и норма определяются равенством\[||f||_{C^(n)}=\displaystyle\sum_{k=0}^n\max_{x\in[a,b]}|f^{(k)}(x)|=\sum_{k=0}^n||f^{(k)}||_C,\]где \(f^{(0)}(x)=f(x)\).

\section{Функционал и ее вариация.}

\par\ 
\par\textbf{Определение.} \textit{Приращением} или \textit{вариацией} функционала \(\delta y\) аргумента \(y\) функционала \(J:E\to\mathbb R\) называется разность между двумя элементами \(y,\tilde y\in E\), т.е. \(\delta y =\tilde y - y\).
\[(\delta y)'=\tilde y' - y' = \delta y',\quad(\delta y)''=\tilde y'' - y'' = \delta y'',\quad...,\quad(\delta y)^{(n)}=\tilde y^{(n)} - y^{(n)} = \delta y^{(n)}.\]
\par\textbf{Определение.} Говорят, что кривые \(y=y(x)\) и \(y_1=y_1(x)\), заданные на отрезке \([a,b]\), \textit{близки} в смысле близости нулевого порядка, если \(|y(x)-y_1(x)|\) мала на \([a,b]\). Геометрически это означает, что эти кривые на отрезке \([a,b]\) близки по ординатам.
\par\textbf{Определение.} \textit{Расстоянием} между кривыми \(y=f(x)\) и \(y=f_1(x),\ x\in[a,b]\), где \(f(x)\) и \(f_1(x)\) -- непрерывные на \([a,b]\) функции, называется неотрицательное число \(\rho\), равное максимуму \(|f_1(x)-f(x)|\) на \([a,b]\):\[\rho=\rho[f_1(x),f(x)]=\displaystyle\max_{x\in[a,b]}|f_1(x)-f(x)|.\]
\parРасстоянием \(n\)-го порядка называется величина\[\rho_n=\rho_n[f_1(x),f(x)]=\displaystyle\max_{0\le k\le n}\max_{x\in[a,b]}|f_1^{(k)}(x)-f^{(k)}(x)|.\]
\par\textbf{Первое определение вариации функционала.} Величину \(\Delta J = \Delta J(\delta y)=J[y+\delta y]-J[y]\) называют \textit{приращением функционала}, отвечающим приращению \(\delta y\). При фиксированном \(y\) приращение \(\Delta J(\delta y)\) представляет собой функционал от \(\delta y\). Предположим, что приращение можно представить в виде суммы
\begin{equation}
    \Delta J=\Delta J(\delta y)=L(\delta y)+o(||\delta y||),
\end{equation}
где \(L(\delta y)\) -- функционал, линейный относительно вариации \(\delta y\), а \(o(||\delta y||)\) -- бесконечно малая более высокого порядка по сравнению с \(||\delta y||\). Главную часть \(L(\delta y)\) этого приращения \(\Delta J\), линейную относительно \(\delta y\), называют \textit{вариацией функционала} \(J[y(x)]\) и обозначают через \(\delta J\), т.е. \(\delta J=L(\delta y)\). В этом случае функционал \(J|y(x)|\) называется дифференцируемым в точке \(y(x)\).
\par\textbf{Второе определение вариации функционала.} Вариацией функционала \(J|y(x)|\) в точке \(y(x)\) называется значение производной функционала \(J[y(x)]\) по параметру \(\alpha\), когда \(\alpha=0\):\[\delta J=\dfrac{\partial}{\partial\alpha}J[y(x)+\alpha\delta y(x)]|_{\alpha=0}.\]
\par\textbf{Определение.} Функционал \(J[y,z]\) называется зависящей от двух элементов \(y,z\), называется \textit{билинейным}, если при фиксированном \(y\) он представляет собой линейный функционал от \(x\), а при фиксированном \(y\) он представляет собой линейный функционал от \(z\), а при фиксированном \(z\) -- линейный функционал от \(y\). Таким образом, функционал \(J[y,z]\) билинеен, если
\begin{align*}
    J[\alpha_1y_1+\alpha_2y_2,z]=\alpha_1J[y_1,z]+\alpha_2J[y_2,z],\\ J[y,\beta_1z_1+\beta_2z_2]=\beta_1J[y,z_1]+\beta_2J[y,z_2].
\end{align*}
Полагая в билинейном функционале \(z=y\), получаем выражение \(J[y,y]\), называемое \textit{квадратичным функционалом}.
\par\textbf{Определение.} Квадратичный функционал называется \textit{положительно определенным}, если \(J[y,y]>0\) для любого ненулевого элемента \(y\). 
\par\textbf{Определение.} Пусть \(J[y]\) -- функционал, определенный в каком-либо линейном нормированном пространстве. Мы скажем, что фукнционал \(J[y]\) имеет вторую вариацию, если его приращение \(\Delta J=J[y+\delta y]-J[y]\) можно записать в виде\[\Delta J=L_1(\delta y)+\dfrac{1}{2}L_2(\delta y)+\beta||\delta y||^2,\] где \(L_1\) -- линейный функционал, \(L_2\) -- квадратичный функционал, а \(\beta\to0\) при \(||\delta y||\to0\).
\parКвадратичный функционал \(L_2(\delta y)\) будем называть второй вариацией (вторый дифференциалом) функционала \(J[y]\) и обозначать \(\delta^2J\).

\section{Экстремум функции.}

\ 
\par\textbf{Определение.} Говорят, что функционал \(J[y(x)]\) достигает \textit{экстремума} на кривой \(y=y_0(x)\), если на любой близкой к \(y=y_0(x)\) кривой приращение \(\Delta J=J[y(x)]-J[y_0(x)]\), если на любой близкой к \(y=y_0(x)\) кривой сохраняет знак, при этом если \(\Delta J\ge0\), причем \(\Delta J=0\) только при \(y=y_0(x)\), то \(J[y(x)]\) достигает \textit{строгого минимума} при \(y=y_0(x)\), если \(\Delta J\le0\), то \textit{строгого максимума}.
\par\textbf{Определение.} Функционал \(J[y],\ y\in E\), достигает экстремума при \(y=y_0(x)\), если существует окрестность точки \(y=y_0(x)\)\[P(y_0,\varepsilon)=\{y\in E, ||y-y_0||<\varepsilon\},\]в которой приращение \(\Delta J=J[y(x)]-J[y_0(x)]\) сохраняет знак, при этом если \(\Delta J>0\), то \(J[y(x)]\) достигает \textit{минимума} при \(y=y_0(x)\), если \(\Delta J<0\), то \textit{максимума}.
\par\textbf{Теорема 1.} Если функционал \(J[y(x)]\) достигает экстремума при \(y=y_0(x)\), то его вариация обращается в ноль при \(y=y_0(x)\), т.е.\[\delta J|_{y=y_0}=0.\]
\par\textit{Доказательство.} Пусть функционал \(J[y(x)]\) задан на нормированном пространстве \(E\) с нормой \(||y||\) и достигает для определенности при \(y=y_0(x)\) минимума. По определению минимума функционала, его приращение \[\Delta J=J[y(x)]-J[y_0(x)]=J[y_0(x)+h(x)]-J[y(x)]>0\]в некоторой \(\varepsilon\)-окрестности точки \(y_0\)\[P(y_0,\varepsilon)=\{y\in E,||y-y_0||=||h||<\varepsilon\},\quad y\neq y_0.\]
По условию теоремы существует вариация, поэтому приращение представляется в виде \[\Delta J=\delta J(h)+o(||h||) > 0.\]
\parПредположим, что вариация \(\delta J(h)\neq0\). Так как величина \(o(||h||)\) бесконечно малая более высокого порядка, чем \(||h||\), то она не влияет на знак приращения и поэтому \[\sign\Delta J(h)=\sign\delta J(h).\]По предположению, \(\Delta J(h)>0\) в \(\varepsilon\)-окрестности точки \(y_0\), следовательно, и \(\delta J(h)>0\) в этой окрестности.
\parЕсли \(y_0+h\) принадлежит \(\varepsilon\)-окрестности точки \(y_0\), и \(y_0-h\) также принадлежит этой окрестности. Из линейности функционала следует, что \(\delta J(-h)=-\delta J(h)\).
\parТаким образом, функционалы \(\delta J(h)\) и \(\delta J(-h)\) имеют разные знаки и прираращение \(\Delta J\) не сохраняет своего знака ни в какой \(\varepsilon\)-окрестности точки \(y_0\). Следовательно, \(y_0\) не может давать минимума функционалу -- противоречие. \(\blacksquare\)

\par\ 

\parВ вариационном исчислении расссматриваются функционалы вида
\begin{equation}
    J[y(x)]=\int_a^bF(x,y,y')dx,
\end{equation}
где \(F(x,y,z)\) -- непрерывная функция, имеющая непрерывные частные производные по всем переменным до второго порядка включительно. Можно показать, что при выполнении условий, наложенных на \(F(x,y,z)\) функционал (1) непрерывен в \(C^{(1)}\). Найдем вариацию этого функционала в пространстве \(C^{(1)}\). Пусть \(\delta y = h(x)\) -- приращение функции \(y(x)\), тогда \(\delta y'=h'\). Применяя к разности подынтегральных функций формулу Тейлора, для приращения \(\Delta J\) получаем соотношение:
\begin{align*}
\Delta J=\displaystyle\int^b_a(F(x,y+h,y'+h')-F(x,y,y'))dx=\int^b_a(F_y(x,y,y')h+F_{y'}(x,y,y')h')dx +\\+ \dfrac{1}{2}\int^b_a(F_{yy}(x,y,y')h^2+2F_{yy'}(x,y,y')hh'+F_{y'y'}(x,y,y')h'^2)dx.
\end{align*}
Второй интеграл в этом равенстве обозначим через \(I_2(h)\) и перепишем \(\Delta J\) в виде\[\Delta J=\displaystyle\int^b_a(F_yh+F_{y'}h')dx+I_2(h).\]
\par\textbf{Основная лемма вариационного исчисления.} Пусть \(\alpha(x)\) -- фиксированная, непрерывная на \([a,b]\) вместе со своей производной функции \(h(x)\) такой, что \(h(a)=h(b)=0\), имеет место равенство
\begin{equation}
    \int_a^b\alpha(x)h(x)dx=0,
\end{equation}
то \(\alpha(x)\equiv0\) на \((a,b)\).
\par\textit{Доказательство.} Предположим, что \(\alpha(x)\neq0\), тогда существует точка \(\xi\in(a,b)\) такая, что \(\alpha(\xi)\neq0\), пусть для определенности \(\alpha(\xi)>0\). По свойству непрерывности, существует интервал \((x_1,x_2)\subset(a,b)\), в котором \(\alpha(x)>0\). Рассмотрим функцию
\begin{equation}
    \tilde h(x)=\left\{
\begin{array}{cc}
     (x-x_1)^{2}(x_2-x)^{2},&  x\in(x_1,x_2)\\
     0,& x\in(x_1,x_2).
\end{array}\right.
\end{equation}
Очевидно, что \(\tilde h(x)\) и \(\tilde h'(x)\) непрерывны на \([a,b]\) и \(\tilde h(a)=\tilde h(b)=0\). Подынтегральная функция \(\alpha(x)h(x)\) больше нуля при \(x\in(x_1,x_2)\), по свойству интеграла знак неравенства сохранится и для интеграла:\[\displaystyle\int_a^b\alpha(x)\tilde h(x)dx=\int_{x_1}^{x_2}\alpha(x)(x-x_1)^2(x_2-x)^2dx>0.\]
\parТаким образом, для выбранной нами функции \(\tilde h(x)\) не выполняется соотношение (2), т.е пришли к противоречию.
\(\blacksquare\)

\par\textbf{Замечание.} Если мы будем рассматривать не \(C^{(1)}\), а \(C^{(n)}\) вместо функции (3) можно рассмотреть функцию\[h(x)=\left\{
\begin{array}{cc}
     (x-x_1)^{2n}(x_2-x)^{2n},&  x\in(x_1,x_2)\\
     0,& x\in(x_1,x_2).
\end{array}\right.\]
\par\textbf{Пример.}
\par1. Вычислить функционал \(J[y(x)]=\displaystyle\int^1_0y(x)dx,\ y\in C[0,1],\quad y(x)=1,\quad y(x)=e^x\).
\par\textit{Решение.} \(J[1]=\displaystyle\int^1_01dx=x|^1_0=1,\quad J[e^x]=\int^1_0e^xdx=e^x|_0^1=e-1.\)
\par2. Найти расстояние между кривыми \(y=x,y=x^2,\ x\in[0,1]\).
\par\textit{Решение.} \(\rho(x,x^2)=\max_{x\in[0,1]}|x-x^2|=\max_{x\in[0,1]}(x-x^2)\) так как на отрезке \([0,1]\) парабола \(y=x^2\) ниже прямой \(y=x\). Пусть \(g(x)=x-x^2,\ g'(x)=1-2x=0\Rightarrow x=\frac{1}{2}\). Вычисляем значение в стационарной точке\(g(\frac{1}{2})=\frac{1}{2}-(\frac{1}{2})^2=1\frac{1}{4}\) и на границах \(g(0)=0,\ g(1)=0 \Rightarrow \rho(x,x^2)=\frac{1}{4}\).

\section{Вариационные задачи с закрепленными концами.}

\ 
\parБудем рассматривать функционал (1)
\begin{equation}
    J[y(x)]=\int^b_aF(x,y,y')dx
\end{equation}
на множестве функций \(y(x)\in C^{(1)}\), удовлетворяющих граничным условиям
\begin{equation}
    y(a)=A,\quad y(b) = B
\end{equation}
\par\textbf{Вариационная задача.} Среди функций \(y(x)\) с граничным условием (2) найти такие, которые дают экстремум функционалу (1).
\par\textbf{Теорема.} Если функция \(y(x)\in C^{(1)}\) удовлетворяет граничным условиям (2) и дает экстремум функционалу (1), то она является решением уравнения \textit{Эйлера}
\begin{equation}
    F_y-\dfrac{d}{dx}F_{y'}=0
\end{equation}
\par\textit{Доказательство.} Воспользуемся формулой \(\delta J=\displaystyle\int_a^b(F_yh+F_{y'}h')dx\).
\parПусть функция \(y(x)\) дает экстремум функицоналу (1). По необходимому условию существования экстремума, вариация функционала должна равняться нулю, т.е
\begin{equation*}
    \delta J=\int_a^b(F_yh+F_{y'}h')dx=0
\end{equation*}
\parИнтегрируем по частям и получаем
\begin{equation*}
    \int_a^bF_{y'}hdx=F_{y'}h|_a^b-\int^b_a\dfrac{d}{dx}F_{y'}\cdot hdx.
\end{equation*}
\parИз граничных условий следует, что рассматриваемые приращением \(h(x)\) должны обращаться в ноль в точках \(x=a\) и \(x=b\), т.е.
\begin{equation}
    h(a)=h(b)=0.
\end{equation}
Тогда получаем
\begin{equation*}
    \delta J=\int_a^b(F_y-\dfrac{d}{dx}F{y'})hdx=0
\end{equation*}
Но функция \(\alpha(x)=F_y-\dfrac{d}{dx}F_{y'}\) непрерывна на \([a,b]\), а \(h(x)\) -- любая функция, непрерывная вместе с первой производной на \([a,b]\) и удовлетворяют условиям \(h(a)=h(b)=0\). Осталось применить основную лемму вариационного исчисления.\(\blacksquare\)
\par\textbf{Определение.} Функции, являющиеся решениями уравнений Эйлера, называют \textit{экстремалями}.
\par\textbf{Теорема.} Пусть \(y(x)\) есть решение уравнения Эйлера (3). Если функция \(F(x,y,y')\) имеет непрерывные частные производные до второго порядка включительно, то во всех точках \((x,y)\), в которых 
\begin{equation*}
    F_{y'y'}(x,yy')\neq0,
\end{equation*}
функция \(y=y(x)\) имеет непрерывную вторую производную.
\par\textbf{Следствие.} Экстремаль \(y(x)\) может иметь излом только в тех точках, где \(F_{y'y'}=0\).
\par\textbf{Теорема Бернштейна.} Пусть имеется уравнение
\begin{equation}
    y''=F(x,y,y').
\end{equation}
Если функции \(F_x, F_y, T_{y'}\) непрерывны в каждой конечной точке \((x,y)\) для любого конечного \(y'\) и если существует такая константа \(k=0\) и такие, ограниченные в каждой конечной части плоскости функции \[\alpha=\alpha(x,y)\ge0,\quad \beta=\beta(x,y)\ge0,\] что \[F_y(x,y,y')>k,\quad|F(x,y,y')|\le\alpha y'^2+\beta,\] то через любые две точки плоскости \((a, A)\) и \((b, B)\), имеющие различные абсциссы, проходит ровно ода интегральная кривая \(y=\varphi(x)\) уравнения (5).
\par\textbf{Теорема.} Пусть функции \(p(x), p'(x), q(x), f(x)\) непрерывны на отрезке \([a,b]\) и \(p(x)>0,\ q(x)\ge0\). Если \(y(x)\) есть экстремаль для функционала
\begin{equation}
    J[y(x)]=\int_a^b[p(x)y'^2+q(x)y^2+2yf(x)]dx
\end{equation}
и удовлетворяет условиям (2), то она реализует абсолютный минимум функционала (6), т.е. для любой другой допустимой кривой \(\tilde y(x)\) выполняется неравенство \(J(\tilde y)>J(y)\).

\section{Вариационные задачи для функций нескольких переменных.}

\ 
\parРассмотрим функцию двух переменных \(z(x,y)\). Тогда \(\delta z = \tilde{z}-z,\ (\delta z)_x=\tilde{z}_x-z_x=\delta z_x=h_x,\ (\delta z)_y=\tilde{z}_y-z_y=\delta z_y=h_y\). Рассматриваем функционал
\begin{equation}
    J[x(x,y)]=\int\int_D F(x,y,z,z_x,z_y)dxdy,
\end{equation}
где \(D\in\mathbb R^2\), с граничным условием
\begin{equation}
    z|_\gamma=\varphi(x,y),
\end{equation}
где \(\gamma\) -- граница области \(D\).
\parСчитается, что область \(D\) не варьируется. Задача с условием (2) геометрически означает, что в плоскости \(xOy\) задана область \(D\) с границей \(\gamma\) и задан пространственный контур \(\Gamma\), проекция которого на плоскость \(xOy\) является кривой \(\gamma\). Все допустимые поверхности \(z(x,y)\) натянуты на \(\Gamma\).
\parДля вариационной задачи (1) (2) имеет место лемма основной вариационной задачи, если фиксированная функция \(\alpha(x,y)\) непрерывна в области \(\overline G\) и для любой непрерывной в замыкании \(\overline{G}\) функции \(h(x,y)\), имеющей непрерывные частные производные до 2-го порядка в области \(G\). \[h|_\gamma=0,\ \int\int_D\alpha(x,y)dxdy=0\Rightarrow\alpha(x,y)=0\ \textup{в области}\ D\].
\par\textbf{Теорема.} Если функция \(z(x,y)\), удовлетворяющее граничным условиям (2) дает экстремум функционалу (1), то она является решением уравнения Эйлера-Остроградского
\begin{equation}
F_z-\dfrac{\partial}{\partial x}F_{z_x}-\dfrac{\partial}{\partial y}F_{z_y}=0
\end{equation}
\par\textit{Доказательство.} Будем искать вариацию. \[\Delta J=\int\int_DF(x,y,z+h,z_x+h_x,z_y+h_y)-F(x,y,z,z_x,z_y)dxdy=\]\[=\int\int_DF+F_zh+F_{z_x}h_x+F_{z_y}h_y+...-Fdxdy=\int\int_DF_zh+F_{z_x}h_x+F_{z_y}h_ydxdy+I_2(h).\] Тогда вариация \(\delta J=\int\int_DF_zh+F_{z_x}h_x+F_{z_y}h_ydxdy\).
\[\dfrac{\partial}{\partial x}(F_{z_x}h)=h\dfrac{\partial}{\partial x}F_{z_x}+F_{z_x}h_x,\ \dfrac{\partial}{\partial y}(F_{z_y}h)=h\dfrac{\partial}{\partial y}F_{z_y}+F_{z_y}h_y.\]
Тогда вариацию можно переписать следующим образом
\[\delta J=\int\int_Dh(F_z-\dfrac{\partial}{\partial x}F_{z_x}-\dfrac{\partial}{\partial y}F_{z_y}dxdy+\int\int_D\dfrac{\partial}{\partial x}(F_{z_x}h)-\dfrac{\partial}{\partial y}(F_{z_y}h)dxdy\]
\[I_2=\oint_\gamma F_{z_x}hdy-F_{z_y}hdx,\ h=\delta z=\tilde z - z,\ \tilde z|_y=\varphi,\ z|_\gamma=\varphi,\ h|_\gamma=0.\]
Таким образом теорема доказана.

\parПусть вариационная задача рассматривается без граничного условия (2). Гоеметрически это означает, что вместо одного пространственного контура мы рассматриваем множество контуров \(\Gamma_i\), которое проецируется в плоскости \(\gamma\) и на которые натянуто \(z(x,y)\).
\parЕсли \(z(x,y)\) дает экстремум функционалу (1) на произвольных поверхностях, то она тем более будет давать экстремум для поверхностей, натянутых на контур \(\gamma\).

\end{document}